\section*{Discussion}
\quot{What is translation? On a platter \\ A poet's pale and glaring
  head}{Vladimir Nabokov}
\paragraph{}{Following the heavy math of the previous section, I
  return here to the crux of the problem: poetry, and translating
  it. To dig into the meat of the problem, let's consider an exemplar
  poem, and how we might approach translating it, both by hand and
  computationally. }

\paragraph{}{Vladimir Nabokov, in the notes of his controversial
  translation of Aleksandr Pushkin's work `Eugene Onegin'
  \cite{nabokov, nabokovtheory}, argues that all translations of poetry will
  inevitably fall under three categories:}
  \begin{description}
  \item [Paraphrastic] A free version of the original: words and
    phrases are toyed with --- added, removed, or changed --- in order
    to conform to some form that the translator wishes.
  \item [Lexical] Translating the basic meaning of words and their
    order.
  \item [Literal] Translating as closely as possible the original,
    with the exact contextual meaning being preserved.
  \end{description}

  \paragraph{}{Nabokov believed that only the final item, literal
    translation, was a `true' translation, and all others only served
    to tarnish the poem:}
  \begin{displayquote}
    The hack who has never read the original, and does not know its
    language, praises an imitation as readable because easy platitudes
    have replaced in it the intricacies of which he is unaware.
  \end{displayquote}
  \paragraph{}{Nabokov's translation of `Eugene Onegin', predictably,
    fell into the `literal' camp of translation. The original poem
    consisted of stanzas of iambic tetrameter, with the rhyme scheme
    `AbAbCCddEffEgg'. Note that uppercase letters represent
      feminine rhymes (a rhyme which matches two or more syllables, in
      which the final syllables are unstressed), and lowercase letters
      represent masculine rhymes (a rhyme with matches only one
      syllable, which is stressed). Nabokov argued that ``to
    reproduce the rhymes and yet translate the entire poem literally
    is mathematically impossible'', and as such decided to eschew
    translating the rhyme scheme: his final version contained no
    rhymes, but translated meticulously every word of vocabulary and
    every structural choice which Pushkin made while writing the poem,
    as well as containing extensive commentary. It is considered the
    most `faithful' translation of the original, and it serves as a
    useful gloss for those wanting to read the original poem, but not
    possessing a suitable level of Russian.  }

  \paragraph{}{With these three options in mind, let's explore the
    manual and computational approaches one may take to translating
    what is considered to be one of the most beautiful works in the
    German language\cite{goethe}: Goethe's `{\"U}ber allen Gipfeln'
  \footnote{Also called `Wanderer's Nightsong II', as it is the
      second part of Goethe's `Wandrers Nachtlied' series. } Here is
    the text of the original: \\}

% Hacky fix, needs to stay on one line
\begin{minipage}{0.8\linewidth}
  \begin{verse}

    {\"U}ber allen Gipfeln \\
    Ist Ruh, \\
    In allen Wipfeln \\
    Sp{\"u}rest du \\
    Kaum einen Hauch; \\
    Die V{\"o}gelein schweigen im Walde. \\
    Warte nur, balde \\
    Ruhest du auch.
  \end{verse}
  \attrib{Johann Wolfgang von Goethe (1749--1832)}
\end{minipage}

\paragraph{}{Firstly, let us consider a lexical translation --- that
  is, a translation which considers only the basic preservation of
  words and their order. The following gloss was done using an
  English-German dictionary, with alternative words marked using
  slashes: \\}

% \begin{minipage}{0.8\linewidth}
  \begin{verse}
    Over all peaks\slash summits\slash tops \\
    Is rest\slash peace\slash silence, \\
    In all treetops \\
    You (informal) sense\slash feel \\
    Hardly a breath\slash breeze; \\
    The little birds remain silent\slash keep still in the wood\slash forest. \\
    Just wait, soon \\
    You rest\slash repose too\slash also. \\
  \end{verse}
  \attrib{TU Chemnitz Dictionary\cite{beo}}
% \end{minipage}

  \paragraph{}{Doing this by hand only requires a bilingual dictionary
    and knowledge of the target language: as such, a computational
    approach would, at most, require access to a good English-German
    dictionary. This is, then, the simplest computational approach,
    requiring no large amounts of mathematics or programming. Note how
    the rhyme scheme has not been preserved, excepting the duplication
    of `tops' in lines 1 and 3,\footnote{While this is technically a
      rhyme, a poet using a word as its own rhyme is committing a
      crime against poetry.} or the half-rhyme of peaks-peace in lines
    1 and 2. Looking at preservation of rhythm is pointless, because
    the translation purposely sacrifices ease of reading for
    completeness of translation: it would not make sense to read this
    poem aloud.\footnote{Unless one is E.E. Cummings, in which case it
      is fine.}}

\paragraph{}{Next, consider what Nabokov would have called the
  paraphrastic version: John Whaley's translation, which hangs on a
  plaque in a reproduction of the cabin where Goethe wrote the
  original poem: \\}

\begin{minipage}{0.8\linewidth}
  \begin{verse}
    Over all of the hills \\
    Peace comes anew, \\
    The woodland stills \\
    All through; \\ 
    The birds make no sound on the bough. \\
    Wait a while, \\
    Soon now \\
    Peace comes to you. \\
  \end{verse}
  \attrib{John Whaley}
\end{minipage}

\paragraph{}{Let us consider which parts of the original poem this
  translation preserves, and which parts it changes. The rhyme scheme
  of the original is {\it ababcddc}; the translation, on the other
  hand, is {\it ababcdcb} --- so the rhyme scheme has been preserved
  to an extent. But to allow for this preservation of rhyme scheme,
  the poem has been altered in a way that would disgust Nabokov. The
  first line is translated in the same way. The second line goes from
  `is peace' to `peace comes anew' --- a change in tense which is only
  done to allow a rhyme two lines later. The rest of the poem has been
  completely altered: it goes from `in all treetops, you feel hardly a
  breath' to `the woodland stills all though'; the birds move from the
  forest to the bough; and instead of you finding peace, the peace
  comes to you.}
\paragraph{}{The original poem is considered one of the most beautiful
  works in the German language, primarily for two reasons. The first
  is the contrast Goethe draws between man and nature: man is
  restless, uncomfortable in the silence of the forest but expecting
  of death's eternal peace. On the other hand, nature is united in its
  silence. The scale of the poem is also interesting: it moves from a
  large scale (the summits), to a middle distance (the treetops), to
  the immediate surroundings (in the forest), and finally to the
  person. This sequence can be seen as encompassing the human
  universe, moving from the largest (visible from a forest) scale, to
  the smallest, most personal. The translation of the poem loses the
  large scale (summits turn to hills), although it preserves the progression
  of scale reduction. But man's restlessness is lost: instead of man
  having to wait for eternal peace, it becomes peace coming to man.}
\paragraph{}{Instead of looking at this poem as a translation of
  Goethe's original, it might be better to look at it as a poetic work
  in its own right. Looking at it from that point of view, it is very
  much a good poem. There is a good rhythm, a good rhyme scheme, and
  it reads well: and, while there is not necessarily the same level of
  conciseness and beauty that Goethe's poem shows, it is still a read
  to be savored.}
\paragraph{}{It is worth discussing here the work of Douglas
  Hofstadter in his book `Le Ton Beau de Marot' \cite{hof}. The book
  covers possible translations of the Cl\'{e}ment Marot's `Ma
  Mignonne': by exploring various possible translations, the author
  explores a wide range of topics in linguistics, psychology, and
  mathematical theory. Hofstadter is perhaps the most vocal supporter
  (outside of translation circles) of the paraphrastic approach to
  translation. The book contains a multitude of possible translations
  and transformations of the poem, most of which are barely
  recognizable to be related to the original. They are good reads,
  just as Whaley's translation is a good read. But they do not stay
  faithful to the original. While intended to be a proof of the
  superiority of the paraphrastic approach to translation, what
  Hofstadter's book really shows us is the unsuitability of
  paraphrastic translation to preserving the context and meaning of
  the poem. Instead, it can be seen as proof that paraphrastic
  translation creates a new, distinct poem.  }

\paragraph{}{How would one go about manually taking the literal
  approach to this poem? I believe the task is beyond the scope of
  this report. The main issue is that there is a deep understanding of
  both languages required: Nabokov himself was bilingual in both
  English and Russian from childhood, and thus had the kind of
  intimate understanding of both languages and cultures which I
  simply do not possess. Additionally, there are time constraints:
  Nabokov's own translation took many years of research
  \cite{nabokov}, and the time frame for this project was
  limited. Perhaps one might start with exploring Goethe's background,
  his other poetry, and so on, and explore the area, working and
  reworking a translation in much the same way as the poet might have,
  until a suitably literal translation is reached.}

\paragraph{}{Considering all of the above, what is the best
  computational approach we could take to translate poetry? Manually,
  the best approach, in my opinion, is the literal approach: a
  preservation of the original poem's contextual meaning. A
  computational literal translation can be considered to be under the
  umbrella of `AI-complete' problems. Paraphrastic translation, for
  all Nabokov disparaged it, is a good computational option: while it
  does not preserve the original, it does create a new poem which, as
  seen above, is still worth reading in its own right: it is also
  easier, computationally, to implement.}

\paragraph{}{To get an idea of what we're working with, here is how
  Google Translate, arguably the most popular translation service
  today, translates the poem.\footnote{It must be noted here that as
    of 2016, Google Translate uses neural machine translation for
    English $\leftrightarrow$ German translation, the explanation and
    analysis of which is beyond the scope of this report. I will be
    focusing only on the types of translation I outlined in the theory
    section; nonetheless, analyzing its poetic output is an
    interesting exercise. }\\}

\begin{minipage}{0.8\linewidth}
  \begin{verse}
    Above all peaks \\
    Is peace, \\
    In all tops \\
    Do you feel \\
    Hardly a breath; \\
    The birds are silent in the forest. \\
    wait, soon \\ 
    Are you retiring too? \\
  \end{verse}
  \attrib{Google Translate}
\end{minipage}

\paragraph{}{The first thing to notice about this translation is its
  somewhat perplexing tendency towards converting statements to
  questions. `Sp{\"u}rest du' becomes `do you feel' --- a reasonable
  mistake to make, because the German language does not use an
  auxiliary (`do') to indicate a question: it simply reverses the verb
  and the noun to indicate its presence. However, the following and
  preceding lines make it clear that this is not an isolated phrase,
  which could be interpreted as a question, but rather part of a
  sentence clause. Another interesting thing is the vocabulary
  choices: `Gipfeln' is translated to `peaks', as opposed to the more
  common translation `summits'; the word `nur' is also cut completely,
  which eliminates the feeling of urgency and restlessness which its
  use in the original poem implies.}

\paragraph{}{My research has consisted of hypothesizing changes to
  current (word- and phrase-based) machine translation methods which
  would allow for more fluent poetic output, compared to the type
  shown above. I have developed three
  hypotheses:}
\begin{enumerate}
\item Training the language and alignment models on a poetic corpus
  improves poetic qualities of the output translation.
\item Altering the sentence alignment model for poetic works will
  improve line-by-line translation quality.
\item Poetic characteristics can be preserved by constraints on the
  hypothesis space, or recovered by post-processing of the output
  translation.
\end{enumerate}

\paragraph{Poetic corpus}{I would argue that statistical translation
  of poetry faces the same problem as statistical modeling of
  language: we simply have no way of truly modeling the poet's thought
  processes while writing the original, and therefore no way of
  emulating it computationally. So, again, we have to compromise:
  instead of seeking to emulate the thought process, we instead seek
  to statistically model poetry in such a way that our translation
  sounds poetical. }
\paragraph{}{The primary way of having a `poetic' sounding translation
  is through corpus choices. Google Translate's language and alignment
  models for English $\leftrightarrow$ German translation are
  primarily trained on the Europarl corpus, which is a transcription
  of all European Parliament proceedings, and is available in all the
  official languages of the Parliament \cite{parl}. Multilingual
  corpus availability is a large problem in machine translation, and
  the Europarl corpus is large enough that it can be used as a basis
  for alignment models when translating the applicable European
  languages. However, alignment models, like language models, are
  sensitive to the input corpus (even with smoothing), and will prefer
  alignments with words which occur often in the corpus. Europarl
  proceedings are, predictably, very dry and not at all poetic. A
  non-poetic input implies a non-poetic output. The obvious solution
  here is to use a parallel corpus of German poems and their English
  translations. Unfortunately, no such corpuses exist, and creation is
  extremely difficult, as very few parallel poetic translations are
  available in easily-processable digital form. In an ideal world, to
  translate a given poet's poem, we'd use a corpus of parallel
  translated poems from the same time period or style. Unfortunately,
  this is not the ideal world, so some compromise is needed. }
\paragraph{}{Using a poetic language model is a much more reasonable
  and achievable solution. A suitable language model can be built, I
  believe, simply by collecting a large amount of poetry in the target
  language --- in this case, English, and using this to train the
  language model. As part of this project, I scraped the top 500
  English language works from Project Gutenberg tagged under `poetry'
  and collated them into a corpus, which could possibly be used to
  train a language model with a toolkit that allows it. However, this
  method does raise some problems.}
\paragraph{}{Kao (2011) \cite{kao} showed that there are certain
  characteristics which occur specifically in professionally-written
  poems. The primary features, for modern poetry, are firstly the
  reduced presence of rhyme and alliteration, secondly the use of
  concrete nouns, and thirdly the use of less common
  words. Preservation of these features is well-paired with
  phrase-based machine translation. The lack of rhyme and alliteration
  means we do not need to find ways to parse for them, which is a
  difficult task due to variations in pronunciation. The increased use
  of concrete nouns can be applied by adding an extra weighing to
  these tokens while training our language model. The use of less
  common words is more difficult, as language models, by design, favor
  more common words. A possible solution would be performing extra
  smoothing on the corpus which, for example, greatly increases the
  probability of less common words and decreases the probability of
  more common words. The problem with this, of course, is that less
  common words will begin to be used where a poet may not necessarily
  use them: for example, substituting `to have' with `to possess' or
  `to retain'. }

\paragraph{Alignment model}{Whereas sentences in prose are clearly
  defined, poetry does not have strong sentence distinctions. In some
  cases, a line break may indicate the end of a sentence: in others, a
  line break may indicate the end of a clause, or even just
  nothing. Compare the following extracts of poems:\\}


\begin{minipage}{\linewidth}
  \begin{verse}
    one's not half two. It's two are halves of one: \\
    which halves reintegrating,shall occur \\
    no death and any quantity;but than \\
    all numerable mosts the actual more
    \attrib{ee cummings\cite{cummings}}
  \end{verse}
\end{minipage}

\begin{minipage}{\linewidth}
  \begin{verse}
    Sunlight pouring across your skin, your shadow \\
    >[0.55\linewidth] flat on the wall. \\
    The dawn was breaking the bones of your heart like twigs. \\
    You had not expected this, \\
    >[0.1\linewidth] the bedroom gone white, the astronomical light \\
    >[0.4\linewidth] pummeling you in a stream of fists.
    \attrib{Richard Siken \cite{siken}}
  \end{verse}
\end{minipage}

\paragraph{}{In the first poem, line breaks have no meaning: aside
  from adding a small pause when the poem is read out loud, they can
  completely be disregarded. There are no traditional sentence
  boundaries.  This is an issue with phrase-based translation, as it
  relies on the existence of aligned sentences, and to have aligned
  sentences, one needs clearly defined sentence boundaries. An option
  for this would be to define sentence boundaries as new lines, rather
  than standard punctuation (. !  ?). However, in this case, it would
  lead to important phrases, such as `than all' being split. In
  theory, this shouldn't be an issue --- as already mentioned,
  phrase-based translation actually performs better when
  non-syntactically correct phrase alignments are allowed --- however,
  with translation of meaning in poetry being so context-dependent
  (see the Google Translate example), it could be severely detrimental
  to translation quality. }
\paragraph{}{The second poem is one in which using line-endings for
  separating sentence units would not work: in this case, parsing for
  traditional sentence endings will, in fact, work, because the line
  breaks have only been introduced due to the stylistic preferences of
  the author. The most striking part of this poem is actually the
  layout on the page, which I have tried my best to reproduce from the
  original. A translation would need to reproduce this indentation as
  well, but this could be done quite easily by parsing for whitespace
  at the start of a line in the original poem and using the same
  amount of whitespace in the translation.\footnote{This would only
    work for systems where features like whitespace are not stripped
    before being input.}}
\paragraph{}{A possible solution to the sentence alignment issue would
  be to split poems into two types:}
\begin{enumerate}
\item Poems whose sentences can be treated in the same way as prose,
  where sentences can be assumed to be marked by standard sentence
  boundaries.
\item Poems whose sentence boundaries are marked by line-endings,
  rather than punctuation.
\end{enumerate}
\paragraph{}{The first feature is already used in common sentence
  alignment tools: the second feature could be implemented relatively
  easily by tagging new sentences on line breaks. }

\paragraph{Preserving poetic qualities}{Now that we have dealt with
  the first two problems, we now face the third: how do we preserve
  poetic characteristics while translating? State-of-the-art systems,
  as mentioned already, do not put any weight on preserving aspects of
  poetry such as rhythm or rhyme. There are two possible ways: either
  hypothesis constraints, or post-processing.}
\paragraph{}{Genzel, Uszkoreit, and Och (2010) \cite{genzel} use
  hypothesis constraints to preserve poetic features such as meter,
  line length (syllable and word), or rhyme. By treating poetic
  features as constraints on the poem --- for example, a haiku has a
  constraint on line length and syllable count --- they implement a
  generic poetic form feature function for phrase-based statistical
  machine translation systems. This is a feasible solution to the
  problem of preserving certain poetic features. However, there is an
  issue in the output: while it preserves poetic features, it does not
  preserve fluency, which is of high concern when it comes to
  translating poetry. Here is an example from the paper of a baseline
  translation, and a `couplet in amphibrachic tetrameter' (stressed
  syllables italicized): \\}
\begin{displayquote}
  A policeman said that three people were arrested and that the
  material is currently being analyzed. 
\end{displayquote}
\begin{displayquote}
  An {\it of}ficer {\it sta}ted that {\it three} were ar{\it re}sted \\
  and {\it that} the e{\it quip}ment is {\it cur}rently {\it test}ed.
\end{displayquote}

\paragraph{}{Another solution, instead of searching inside the
  hypothesis space, would be to use post-processing. In many uses of
  machine translation, the output already undergoes post-processing by
  experienced translators or native speakers to fine-tune the output.
  We could, in much the same way, recover poetic features which were
  `lost in translation' by computational post-processing. For example,
  if we wanted to preserve the rhyme scheme of the original poem, it
  could be enough to look for the rhyming synonyms of words at the end
  of lines, and replace those words with their synonyms until the
  lines rhyme in the same way as the original. This is a very
  na{\"i}ve approach, but I believe it could work, given a good enough
  synonym dictionary. There could also be a distortion model, in much
  the same way as phrase-based translation systems, to allow for word
  reordering. This could essentially be seen as implementing a second
  machine translation system, but instead of translating from a
  foreign language to English, we translate from non-poetic to poetic
  English. }

\paragraph{}{What I have assumed while discussing the two previous
  methods is that we already know what poetic features a poem
  has. While it is relatively easy for a trained human to identify and
  annotate such features, it is a non-trivial task for computers to do
  perfectly. Let us briefly consider the two main qualities we want to
  preserve: rhythm and rhyme. }
\paragraph{}{Byrd and Chodorow (1985) \cite{byrd} present a method for
  finding the stress patterns and rhyme-endings of unknown words from
  a pre-existing dictionary. For a given word, three things are looked
  at: the words surrounding it alphabetically, the words which are
  likely to rhyme with it, and words with similar word-end
  spelling. This method, however, relies on a good-quality, up-to-date
  dictionary for the target and source languages. For less common
  languages, such a dictionary is very unlikely to
  exist. Additionally, slant rhymes and changes in pronunciation mean
  that whether two words rhyme is not necessarily a yes-or-no
  question.}
\paragraph{}{Reddy and Knight (2014) \cite{reddy} continue the work of
  Greene et al. (2010) \cite{greene} to describe a method of
  unsupervised discovery of rhyme schemes which is very
  promising. Machine learning methods are used to infer the presence
  of rhyme schemes from corpuses of poetry. The major benefit of this
  is that it is language-independent: so long as one has some sort of
  corpus of rhyming poetry in a given language, it is feasible that
  some rhyme schemes are able to be automatically learned by the
  algorithm.}
\paragraph{}{Luckily, the problem of identifying word stress is more
  easily solved. Due to advances in the field of speech
  synthesis\cite{williams, liber}, as well as previous research into
  stress-timing and syllable-timing \cite{dauer}, finding word stress
  is relatively trivial. }
%%% Local Variables:
%%% mode: latex
%%% TeX-master: "Report"
%%% End:
