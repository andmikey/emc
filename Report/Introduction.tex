\section*{Introduction}
\quot{Poetry is that which is lost in translation}{Robert Frost}

\paragraph{}{Translation is hard. It is practically impossible to
  translate long sections of text from one language to another without
  at least some cultural awareness of both languages. If you were to
  leave a monolingual English speaker in a room with an
  English-Chinese dictionary and, say, a copy of the {\it W\v angchu\=
    an j\'i} (a collection of Tang-dynasty poetry by Wang Wei and Pei
  Di, from around 740), and ask them to translate it, the translation
  you will get is unlikely to be any good. Even experienced
  translators and poets struggle with poetry translation; they will
  often misinterpret meaning, change the style or tone of the poem, or
  in some other way alter it to their liking. \cite{wangwei}
  Translation is a fickle mistress, and not for the faint-hearted. }

\paragraph{}{Machine translation is, of course, even harder. In fact,
  machine translation is considered AI-complete.\footnote{Solving a
    problem which is AI-complete is, in practice, the same as managing
    to create an artificial intelligence which can do anything a human
    can.} Translating non-fiction texts (newspapers, etc.) is hard
  enough: sparse data, ambiguity, and insufficient computing resources
  mean that even state-of-the-art systems cannot quite reach perfect
  translation quality. When it comes to translating poetry, the
  stumbling blocks are even higher. Classical machine translation
  systems focus (quite rightly) on preserving the content of text
  above all else\footnote{Interestingly, some professional
    translators do the same --- Eugin Nida, for example, says: ``Only
    rarely can one reproduce both content and form in a translation,
    and hence in general the form is usually sacrificed for the sake
    of the content.''\cite{eugin}} --- in essence, translating only the
  top level of meaning. The small parts of text --- vocabulary
  choices, word order, metaphor --- are invariably `lost in
  translation' when machine translation is used. But it is these small
  details that make poetry as beautiful as it is; content is tightly
  interwoven with form, and translating only the top level of meaning
  may end up leaving the poem completely nonsensical.}

\paragraph{}{This research does not intend to fully solve the machine
  translation problem: in fact, I doubt it will ever be solved. Poetry
  is at its core a human expression, and machines cannot hope to
  emulate that. Discussing the philosophical issues behind machine
  translation is beyond the scope of this report: I will leave those
  issues to the philosophers. Instead, I hope for this report to
  suggest ideas which may act as an aid to human translators, not
  replace them. }

\paragraph{}{This report will be split into three parts. As a
  preliminary, I will explain the history of both computational
  linguistics and machine translation, to give the reader an overview
  of the two tightly woven fields and a historical context for this
  research; there is also a discussion on {\it why} one would want to
  translate poetry computationally. Secondly, I will describe the
  current techniques in machine translation: I will cover the
  statistical modeling of language, and current approaches to machine
  translation, with a focus on word- and phrase-based statistical
  methods. Finally, using Goethe's ``{\"U}ber allen Gipfeln'' I will
  evaluate the current approaches to manual and computational
  translation, and discuss my research into improving these methods. }

%%% Local Variables:
%%% mode: latex
%%% TeX-master: "Report"
%%% End:
