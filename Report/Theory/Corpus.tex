\paragraph{Corpuses}{But before we can start, we need to acquire a
  corpus. A corpus, roughly speaking, is just a collection of text in
  our target language. It can come from any source, and indeed corpora
  used in linguistics vary widely: examples include the British
  National Corpus (consisting of 100 million words, covering British
  English in the late 20th century), Brown Corpus (1 million words,
  and tagged for parts-of-speech), and Google Books N-Gram corpus
  (multilingual, and searchable online).}
\paragraph{}{Within machine translation, we need two types of corpus:
  a monolingual (language model) corpus, and a parallel bilingual
  corpus. When looking for a corpus to use, the most important thing
  to consider is the domain and topic of the text. For machine
  translation, the domain and topic of the corpus should, as closely
  as possible, match the domain and topic of the text to be
  translated. So, for example, if we were to be translating scientific
  documents, then the best choice for a corpus to learn from would be
  a corpus of other scientific documents.  In an ideal world, to
  translate a given poem, we'd have: a monolingual corpus, entirely
  consisting of poems in the target language in the same style as our
  poem to translate; and a bilingual corpus of poems in the source
  language, of the same style as our input poem, translated by
  professionals into the target language. However, that does throw up
  some problems, which will be discussed later. }

\paragraph{Sentence alignment}{Once we have acquired a parallel corpus
  to work with, we still need to perform sentence alignment. When a
  human translator translates some text, they do not necessarily
  translate it word-for-word or sentence by sentence: this is
  especially true for languages where the writing system does not
  distinguish sentence boundaries. By aligning sentences within a
  corpus, we can more easily find word and phrase pairs when building
  our alignment model.}

\paragraph{}{Formally, \cite{smt} given a set of sentences in a
  foreign language $f = f_1, f_2 \dots f_{n_{f}}$ and their
  translations into English $e = e_1, e_2 \dots e_{n_{e}}$, we want to
  find a sentence alignment $S = \{ s_1, s_2 \dots s_n\}$ such that
  all sentences in both the source and target language are present
  within the alignment, and each sentence only occurs in one sentence
  pair. The simplest way to do this is with the Gale-Church sentence
  alignment algorithm. \cite{gale} The intuition behind their
  algorithm is that longer sentences in one language tend to be
  translated to longer sentences in the other language, and
  vice versa. A probabilistic score is assigned to sentence pairs
  based on both the ratio of character lengths in the sentences, and
  the variance of this ratio. Then a dynamic programming framework is
  used to find the most likely sentence alignment.}
\paragraph{}{Two steps are required for this algorithm: first,
  paragraphs of parallel text are aligned, then sentences within each
  paragraph are aligned. Paragraph alignment is reasonably easy, as
  paragraphs are almost always already aligned in translated text, and
  tend to be very clearly marked out. This means that they can be
  aligned automatically, with the output being corrected by hand if
  needed.}

\paragraph{}{First we define a distance function $d$ which allows for
  sentence insertion, deletion, and substitution, which takes four
  arguments, $x_1, y_1, x_2, y_2$. We let:\cite{gale}}
\begin{enumerate}
\item $d(x_1, y_1, 0, 0)$ be the cost of substituting $x_1$ with
  $y_1$,
\item $d(x_1, 0, 0, 0)$ be the cost of deleting $x_1$,
\item $d(0, y_1, 0, 0)$ be the cost of inserting $y_1$,
\item $d(x_1, y_1, x_2, 0)$ be the cost of combining $x_1$ and $x_2$
  to make $y_2$,
\item $d(x_1, y_1, 0, y_2)$ be the cost of expanding $x_1$ to $y_1$
  and $y_2$
\item $d(x_1, y_1, x_2, y_2)$ be the cost of merging $x_1$ and $x_2$
  and matching with $y_1$ and $y_2$
\end{enumerate}

\paragraph{}{Then we can use a dynamic programming algorithm to align
  the sentences, defined as follows:\cite{gale}}
\begin{displayquote}
  Let $s_i$, $i=1 \dots I$, be the sentences of one language, and
  $t_j$, $j=1 \dots J$ be the translations of those sentences in the
  other language. Let $d$ be the distance function defined previously,
  and let $D(i, j)$ be the minimum distance between sentences
  $s_1 \dots s_i$ and their translations $t_1 \dots t_j$, under the
  maximum likelihood alignment. $D(i, j)$ is computed by minimizing
  over six cases (substitution, deletion, insertion, contraction,
  expansion, and merger) which, in effect, impose a set of slope
  constraints. That is, $D(i, j)$ is defined by the following
  recurrence with the initial condition $D(i, j) = 0$:\\
  $$
  D(i, j) = \text{min}
  \begin{cases}
    D(i, j-1)   &+ \quad d(0,   t_j, 0,       0)       \\
    D(i-1, j)   &+ \quad d(s_i, 0,   0,       0)       \\
    D(i-1, j-1) &+ \quad d(s_i, t_j, 0,       0)       \\
    D(i-1, j-2) &+ \quad d(s_i, t_j, 0,       t_{j-1}) \\
    D(i-2, j-1) &+ \quad d(s_i, t_j, s_{i-1}, 0)       \\
    D(i-2, j-2) &+ \quad d(s_i, t_j, s_{i-1}, t_{j-1}) \\    
  \end{cases}
  $$\\
\end{displayquote}
%%% Local Variables:
%%% mode: latex
%%% TeX-master: "../Report"
%%% End:
