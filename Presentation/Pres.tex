%\documentclass[notesonly, handout]{beamer}
\documentclass[notes]{beamer}
\usepackage{verse}
\usepackage{pgfpages}
% \pgfpagesuselayout{4 on 1}[a4paper, border shrink=5mm, landscape]
\usetheme{Dresden}

\newcommand{\quot}[2]{
    \begin{exampleblock}{}
      #1
      \vskip5mm
    \hspace*\fill{\small --- #2}
  \end{exampleblock}
}
\newcommand{\attrib}[1]{%
  \nopagebreak{
    \begin{flushright}
    \footnotesize
    #1
    \end{flushright}
    \par
  }}

\title{Towards Automated Translation of Poetry}
\author{Michael A.}
\institute{Exeter Mathematics School}
\date{March 30, 2017}

\begin{document}
\begin{frame}
  \titlepage
\end{frame}
\begin{frame}
  \tableofcontents
\end{frame}
\section{Introduction}
\begin{frame}{What is this?}
  \begin{itemize}
  \item Translation is hard --- for humans and computers
  \item Poetry is awesome, but difficult to move across languages
  \item What if a computer could translate poetry for us?
  \item EMC: hypotheses for improving existing methods
  \item A look at literary and computational approaches to poetry
  \end{itemize}
\end{frame}

\begin{frame}{What is poetry?}
  \quot{If I read a book and it makes my whole body so cold no fire
    can warm me I know that is poetry. If I feel physically as if the
    top of my head were taken off, I know that is poetry. These are
    the only way I know it. Is there any other way?}{Emily Dickinson}

\end{frame}
\note{
  \begin{itemize}
    \item Quite a qualitative description of poetry
    \item As an expression of feeling, emotion --- just a human thing?
    \item How can machines translate it if it's exclusive to humans?
    \item Does this exclude certain kinds of poetry? What about poems
      that eg. only play with the language, as opposed to seeking to
      be as ground-breaking as Dickinson suggests?
  \end{itemize}
}

\begin{frame}{What is poetry?}
  \quot{Poetry is that which is lost in translation.}{Robert Frost}

\end{frame}
\note{
  \begin{itemize}
    \item Different idea of poetry --- focus on aesthetic and rhythmic form
    \item Emotional response to a poem closely linked to cultural
      context around it --- cannot translate that. Form, on the other
      hand...
    \item More what most people would associate poetry with ---
      rhythm, rhyme, alliteration, form...
    \item EMC was looking at how we can translate these kinds of form,
      rather than emotion
  \end{itemize}
}

\begin{frame}{What is MT?}
  \begin{itemize}
  \item {\bf Machine translation} (MT) is translation using computers
  \item Warren Weaver's theories:
    \begin{enumerate}
    \item Looking at $n$ surrounding words to disambiguate word meaning
    \item ``Deducing any legitimate conclusion from a finite set of
      premises''
    \item Cryptographic process
    \item Descend ``down to the common base of human communication''
    \end{enumerate}
  \end{itemize}
\end{frame}

\begin{frame}{Why translate poetry?}
  \begin{enumerate}
  \item Translated poems contribute massively to English literature
  \item Translating to/from languages with little or no translation
    communities
  \item Translated poems worth reading in their own right
  \item Computational, quantitative treatment of poetry
  \end{enumerate}
\end{frame}


\section{Literary background}
\begin{frame}{Nabokov}
  \quot{The hack who has never read the original, and does not know
    its language, praises an imitation as readable because easy
    platitudes have replaced in it the intricacies of which he is
    unaware.}{Vladimir Nabokov}
\end{frame}
\begin{frame}{Nabokov's theories}
  \begin{itemize}
  \item {\bf Paraphrastic} A free version of the original
  \item {\bf Lexical} Basic translation of words and their order
  \item {\bf Literal} Preserving exact contextual meaning
  \end{itemize}
  \quot{To reproduce the rhymes and yet translate the entire poem is
    computationally impossible.}{Vladimir Nabokov}
\end{frame}
\note{
  \begin{itemize}
  \item Nabokov wrote a translation of Eugene Onegin, which was
    literal --- didn't translate rhyme scheme, but otherwise
    completely faithful to original
  \item However literal impossible to do computationally --- would
    need general AI (cultural context, emotion, etc)
  \item Lexical good for getting general idea of poem, easiest to do
    computationally
  \item Paraphrastic not great - doesn't preserve original poem. But
    doable computationally

  \end{itemize}
}
\begin{frame}{Goethe: original}
  \begin{verse}

    {\"U}ber allen Gipfeln \\
    Ist Ruh, \\
    In allen Wipfeln \\
    Sp{\"u}rest du \\
    Kaum einen Hauch; \\
    Die V{\"o}gelein schweigen im Walde. \\
    Warte nur, balde \\
    Ruhest du auch.
  \end{verse}
  \attrib{Johann Wolfgang von Goethe}
\end{frame}
\note{
  \begin{itemize}
  \item Widely considered most beautiful poem in German language
  \item Contrast between man --- restless, uncomfortable in the
    silence of the forest --- while nature is united in silence
  \item Scale of poem --- large scale (summits), middle distance
    (treetops), immediate surroundings (forest), then finally man
  \item Encompassing all the universe
  \item Stylistically --- note rhyme scheme, meter
  \end{itemize}
}

\begin{frame}{Goethe: lexical}
  \begin{verse}
    Over all peaks\slash summits\slash tops \\
    Is rest\slash peace\slash silence, \\
    In all treetops \\
    You (informal) sense\slash feel \\
    Hardly a breath\slash breeze; \\
    The little birds remain silent\slash keep still in the wood\slash forest. \\
    Just wait, soon \\
    You rest\slash repose too\slash also. \\
  \end{verse}
  \attrib{TU Chemnitz Dictionary}
\end{frame}
\note{
  \begin{itemize}
  \item I've let alternative translations be represented by slashes -
    clearly, many options on how to translate each word
  \item Purposely sacrificing ease of reading, beauty, etc for
    completeness of translation
  \end{itemize}
}
\begin{frame}{Goethe: paraphrastic}
  \begin{verse}
    Over all of the hills \\
    Peace comes anew, \\
    The woodland stills \\
    All through; \\ 
    The birds make no sound on the bough. \\
    Wait a while, \\
    Soon now \\
    Peace comes to you. \\
  \end{verse}
  \attrib{John Whaley}
\end{frame}
\note{
  \begin{itemize}
  \item What it preserves --- rhyme scheme, in a way
  \item Reordering of lines, change of tense to force rhymes
  \item Loses largest scale (summits to hills, again to force a rhyme)
    --- but progression of scale change is preserved
  \item But as a poem on its own, it reads well
  \item Completely doable to do a translation in this style
    computationally
  \end{itemize}
}
\section{Computational background}
\begin{frame}{Modeling language}
  \begin{itemize}
  \item Cognitive models --- Chomsky
  \item Statistical models as an approximation
  \item Monolingual and bilingual corpuses
  \end{itemize}
\end{frame}
\note{
  \begin{itemize}
  \item Humans are good at learning languages, but why?
  \item Chomsky's theory --- `universal grammar', innate language
    faculty that `knows' the rules of language
  \item But exactly {\it what} it is eludes us, so can't model it
    computationally
  \item Instead, approximate it --- mimic it in such a way that it
    seems close to indistinguishable from the real thing
  \item Best to do this using a statistical model --- given a large
    sample of text, should be able to estimate the probability of a
    given word or sentence, and choose the highest probability option
  \item In MT, need to estimate both the probability of words in one
    language, and probability of words being translated from one
    language to another
  \end{itemize}
}
\begin{frame}{Statistical MT}
  By Bayesian decomposition:
  \begin{equation}
    \Pr(e|f) = \frac{\Pr(e)\Pr(f|e)}{\Pr(f)}
  \end{equation}
  $\Pr(f)$ remains constant, so:
  \begin{equation}
    \Pr(e|f) = \Pr(e) \Pr(f|e)
  \end{equation}
  \begin{exampleblock}{}
    \begin{tabular}{l l}
      {\bf Language model} & $\Pr(e)$ \\
      {\bf Alignment model} & $\Pr(f|e)$ \\
    \end{tabular}
  \end{exampleblock}
\end{frame}
\note{
  \begin{itemize}
  \item Want to translate a foreign sentence $f$ into an English
    sentence $e$
  \item Want to maximize the probability that a chosen $e$ is a
    correct translation of $f$ (LHS)
  \item Intuitively --- probability that a translator will produce $e$
    given $f$
  \item LHS is hard to model --- use Bayes' rule to decompose into
    RHS. Denominator is same for all $e$, so simplify equation
  \item Gives two new probability distributions to model --- language
    model and alignment model
  \item Language model makes output sound fluent
  \item Alignment model makes sure foreign words are translated to
    correct English words
  \end{itemize}
}

\section{Hypotheses}
\begin{frame}{Hypotheses}
  \begin{block}{1. Language model}
    Training the language and alignment models on a poetic corpus
    improves poetic qualities of output translation.
  \end{block}
  \begin{block}{2. Sentence alignment}
    Altering sentence alignment for poetic works will improve
    line-by-line translation quality.
  \end{block}
  \begin{block}{3. Post-processing}
    Poetic characteristics can be recovered by post-processing of the
    output translation.
  \end{block}
\end{frame}

\begin{frame}{Language model}
  \begin{itemize}
  \item Estimate the probability of a sequence of words
    $W = w_1, w_2, \dots w_n$ by Markov assumption:
       $$ \Pr(w_n | w_1 \dots w_{n-1}) \approx \Pr(w_n|w_{n-m} \dots
       w_{n-1})$$
     \item Hypothesis: look to choose more poetic corpus
  \end{itemize}
\end{frame}
\note{
  \begin{itemize}
  \item Language model used to make output sound fluent --- want most
    statistically likely output
  \item To estimate probability of a sequence of words: Markov
    assumption, probability of last word is only affected by
    probability of some history of previous words
  \item If we want poetic sounding output, makes sense that we'd need
    poetic sounding input --- for Euro languages, EuroParl tends to be
    used. Not poetic!
  \item Alignment model and language model could both be used here ---
    but we need bilingual corpus for alignment model, which is hard to
    get
  \item So... I built a monolingual corpus myself for the language
    model. Scrape Project Gutenberg.
  \end{itemize}
}

\begin{frame}{Sentence alignment}
  % Definitely some pretty picture here
  \begin{verse}
      one's not half two. It's two are halves of one: \\
      which halves reintegrating,shall occur \\
      no death and any quantity;but than \\
      all numerable mosts the actual more
    \attrib{ee cummings}
  \end{verse}
\end{frame}
\begin{frame}{Sentence alignment}
  \begin{verse}
    Sunlight pouring across your skin, your shadow \\
    >[0.55\linewidth] flat on the wall. \\
    The dawn was breaking the bones of your heart like twigs. \\
    You had not expected this, \\
    the bedroom gone white, the astronomical light \\
    >[0.2\linewidth] pummeling you in a stream of fists.
    \attrib{Richard Siken}
  \end{verse}
\end{frame}
\note{
  \begin{itemize}
  \item Sentence alignment needed for statistical MT --- generally
    sentences will be translated roughly to each other
  \item Sentence boundaries in prose clearly defined --- full stops,
    question marks, etc
  \item But in poetry, line breaks can indicate end of sentence, too
  \item Hypothesis: split into two types of poems, and align like that
  \end{itemize}
}

\begin{frame}{Post-processing}
  \begin{itemize}
  \item Recover poetic characteristics by post-processing
  \item Non-poetic $\rightarrow$ poetic English translation system
  \item But how to identify poetic characteristics to preserve?
  \end{itemize}
\end{frame}
\note{
  \begin{itemize}
  \item Genzel et al. used hypothesis constraints to limit output
    translations to those with eg. certain meter
  \item Solution could be simpler --- implement MT system from
    non-poetic to poetic
  \item Identifying characteristics: some work on finding stress from
    dictionaries and speech synthesis, rhyme from unsupervised learning
  \end{itemize}
}
\section{Conclusion}
\begin{frame}{Where to next?}
  \begin{itemize}
  \item Implementation of hypotheses --- Moses? GIZA?
  \item Applying to other MT systems --- neural networks \\(Google Translate)
  \item Preserving contextual content, emotions
  \end{itemize}
\end{frame}
\begin{frame}
  \quot{Poetry is that which is lost in translation, \\ Unless we use
    a computational calculation.}{Soon?}
\end{frame}

\begin{frame}{Want to know more?}
  \begin{itemize}
  \item Report and poster in Einstein
  \item \url{blog.andm.io/post/emc}
  \item  \url{mandrejczuk@protonmail.com}
  \end{itemize}
\end{frame}
\end{document}

