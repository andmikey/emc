\paragraph{}{Vladimir Nabokov, in the notes of his controversial
  translation of Aleksandr Pushkin's work `Eugene Onegin'
  \cite{nabokov}, argues that all translations of poetry will
  inevitably fall under three categories:}
\begin{description}
\item [Lexical] Translating the basic meaning of words and their
  order.
\item [Paraphrastic] A free version of the original: words and
  phrases are toyed with --- added, removed, or changed --- in order
  to conform to some form that the translator wishes.
\item [Literal] Translating as closely as possible the original, with
  the exact contextual meaning being preserved.
\end{description}

\paragraph{}{Computationally speaking, the easiest to implement is
  lexical translation. At most, it requires a good dictionary
  between the two chosen languages, along with a possible word
  realignment function. It will not preserve rhyme schemes,
  vocabulary choices, or poetic meter: but it may provide a useful
  gloss of the poem for someone only vaguely familiar with the
  original language.}
\paragraph{}{A literal approach, on the other hand, is
  currently impossible. It would require a `true' machine
  translation system: one which is fully able to understand the
  choices the author made, and is able to translate these choices
  into the target language. At this time, a literary translation of
  any poem is an AI-complete problem, and thus unsolvable until a
  strong AI system is built --- that is, a machine which is as
  emotionally intelligent as a person.}
\paragraph{}{However, a paraphrastic translation is entirely within
  the scope of existing machine translation systems. My research has
  concerned developing hypotheses which would allow current
  (phrase-based) translation algorithms to make a reasonable attempt
  at a paraphrastic translation. }
%%% Local Variables:
%%% mode: latex
%%% TeX-master: "Poster"
%%% End:
