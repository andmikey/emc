\documentclass[a1, plainboxedsections, 13pt]{sciposter}
\usepackage[superscript]{cite}
\usepackage{multicol}
\usepackage{blindtext}
\usepackage{csquotes}
\usepackage{amsmath}
\renewcommand{\familydefault}{\rmdefault}

% \linespread{1.2}

\begin{document}
\title{{\Huge Towards Automated Translation of Poetry}}
\author{{\Large Michael A.}}
\maketitle
\conference{Exeter Mathematics Certificate 2017, Year 13 Project}

\begin{multicols}{3}
  \section*{Introduction}
  \paragraph{}{Translating poetry is hard. Without an intimate knowledge
  of the source and target language, a translation of any text will,
  without a doubt, be lacking. Translating poetry is even harder: and
  to do it computationally, almost impossible. This research project
  explores how we may improve current computational approaches to
  allow for better-quality translations of poetry.}
\paragraph{}{Why would we want to translate poetry? A couple of
  reasons come to mind, which specifically motivated my interest in
  this problem:}
\begin{itemize}
\item Foreign-language poems translated to English have historically
  built up the rich tapestry of English literature: where would we be
  without Dante, Virgil, Homer?
\item Machine translation could be used to translate poems to and from
  languages which have small, if any, translation communities.
\item A faithful translation requires the translator to have intimate
  familiarity with both the source and target languages: a computer
  could do one of these jobs, allowing for a lighter cognitive load on
  the translator and perhaps a better-quality translation.
\item Translated poems are often worth reading for their own merits,
  and it would be an interesting creative exercise to analyze machine
  translated poems.
\item The hope that a more computational, quantitative treatment of
  poetry and poetic analysis will encourage the more
  scientifically-minded to explore poetry and the beauty contained
  therein.
\end{itemize}
%%% Local Variables:
%%% mode: latex
%%% TeX-master: "Poster"
%%% End:

  \section*{Literary theories}
  \paragraph{}{Vladimir Nabokov, in the notes of his controversial
  translation of Aleksandr Pushkin's work `Eugene Onegin'
  \cite{nabokov}, argues that all translations of poetry will
  inevitably fall under three categories:}
\begin{description}
\item [Lexical] Translating the basic meaning of words and their
  order.
\item [Paraphrastic] A free version of the original: words and
  phrases are toyed with --- added, removed, or changed --- in order
  to conform to some form that the translator wishes.
\item [Literal] Translating as closely as possible the original, with
  the exact contextual meaning being preserved.
\end{description}

\paragraph{}{Computationally speaking, the easiest to implement is
  lexical translation. At most, it requires a good dictionary
  between the two chosen languages, along with a possible word
  realignment function. It will not preserve rhyme schemes,
  vocabulary choices, or poetic meter: but it may provide a useful
  gloss of the poem for someone only vaguely familiar with the
  original language.}
\paragraph{}{A literal approach, on the other hand, is
  currently impossible. It would require a `true' machine
  translation system: one which is fully able to understand the
  choices the author made, and is able to translate these choices
  into the target language. At this time, a literary translation of
  any poem is an AI-complete problem, and thus unsolvable until a
  strong AI system is built --- that is, a machine which is as
  emotionally intelligent as a person.}
\paragraph{}{However, a paraphrastic translation is entirely within
  the scope of existing machine translation systems. My research has
  concerned developing hypotheses which would allow current
  (phrase-based) translation algorithms to make a reasonable attempt
  at a paraphrastic translation. }
%%% Local Variables:
%%% mode: latex
%%% TeX-master: "Poster"
%%% End:

  \section*{Hypotheses}
  \paragraph{}{I have developed three hypotheses regarding how current
  phrase-based statistical approaches could be improved to better
  translate poetry:}
\begin{enumerate}
\item Training the language and alignment models on a poetic corpus
  improves poetic qualities of output translation.
\item Altering the sentence alignment model for poetic works will
  improve line-by-line translation quality.
\item Poetic characteristics can be preserved by constraints on the
  hypothesis space, or recovered by post-processing of the output
  translation.
\end{enumerate}

%%% Local Variables:
%%% mode: latex
%%% TeX-master: "Poster"
%%% End:

  \section*{Statistical methods}
  \paragraph{A statistical model of language}{All these
  rule-based, computational models of cognitive grammar are bound to
  be incomplete: simply put, we do not yet have enough knowledge of
  how the brain interprets and creates language --- what internal
  structures it uses, and so forth --- to be able to express these
  methods and structures computationally. So we compromise. Natural
  language processing and generation (of the kind needed in machine
  translation) doesn't require fully emulating human use of language:
  it only needs to approximate it, mimic it in such a way that it
  seems close to indistinguishable from the real thing. A good way of
  doing this is a probabilistic approach.}
\paragraph{}{The best way to explain the probabilistic model is in
  context: so let's take a look at the statistical model of machine
  translation. Say we want to translate a foreign sentence $f$ into an
  English sentence $e$. Then, by a probabilistic model\cite{smt}, we
  want our translation engine to maximize the probability that a given
  English sentence is the correct translation of the foreign sentence:
  that is, maximize $ \Pr(e|f) $. Intuitively, we can interpret this
  probability as the probability a translator will produce $e$ in the
  target language when presented with $f$ in the foreign
  language.\cite{ibm}}
\paragraph{}{Modeling this probability distribution is difficult: we
  can simplify the problem by applying Bayes' theorem\cite{bayes}: by
  Bayesian decomposition, we can rewrite $\Pr(e|f)$ as:
  $$ \Pr(e|f)=\frac{\Pr(e)\Pr(f|e)}{\Pr(f)}$$ Note that the
  denominator $\Pr(f)$ does not depend on $e$, and so is the same for
  all possible values of $e$: therefore we can just simplify our
  equation to be: $$ \Pr(e|f)= \Pr(e)\Pr(f|e)$$ This gives us two new
  variables to maximize: $\Pr(e)$, the {\it language model}, and
  $\Pr(f|e)$, the {\it alignment model}.}

%%% Local Variables:
%%% mode: latex
%%% TeX-master: "../Report"
%%% End:

  \section*{Alignment model}
  
 \paragraph{Alignment model}{How we approach our alignment model
   depends on if we are using {\it word-based} or {\it phrase-based}
   machine translation.\cite{smt}}
 \paragraph{Word-based}{While word-based models are no longer
   state-of-the-art, the methods which are used for them remain
   relatively applicable to modern MT problems, as well as being
   interesting in their own right. The following section will be based
   on the IBM models of machine translation first presented in 1990
   \cite{ibm}, which are still used to an extent in the GIZA and
   GIZA\texttt{+}\texttt{+} toolkits for alignment. Word-based models,
   are, predictably, based on translating words in isolation, or {\it
     lexical translation}. Firstly, we take a corpus of texts in a
   foreign language, and their translations into English, and then
   calculate statistics (a lexical translation probability
   distribution) from these.}
 \paragraph{}{Formally speaking, we want to calculate the probability
   that a given foreign word $f$ translates into a given English word
   $e$, given the data in our corpus. The most straightforward way of
   doing this is, again, maximum likelihood estimation: we take the
   count of the amount of times we see $f$ translated to $e$, and
   divide it by the amount of times we see $f$ overall in the
   corpus. Smoothing can again be used here to account for data
   sparsity and OOV tokens.}
 \paragraph{}{Given this probability distribution, we can now get to
   the task of building our alignment model. A foreign sentence $F$
   can be translated to an English sentence $E$ by use of an
   alignment: a mapping of foreign words to English words. We
   can map this using an alignment function, which may allow for
   adding, removing, and duplicating words: in this case, each English
   word must be linked to exactly one input word, which may be a
   special NULL token. On the other hand, one input word may be linked
   to multiple output words, or none at all. \cite{smt}}
 \paragraph{}{The IBM Model 1 is a generative model (meaning it breaks
   the problem up into smaller steps, and then generates a final
   answer using these intermediate steps), and is based only on the
   aforementioned lexical translation probability distribution. It is
   described in quite some detail in Brown et al. (1993) \cite{m1}, so
   I will only give a superficial gloss. In short, the IBM Model 1
   defines the probability of translating a foreign sentence
   $f = f_1, f_2 \dots f_n$ of length $l_f$ into an English sentence
   $e = e_1, e_2 \dots e_n$ of length $l_e$, with an alignment of each
   English word $e_j$ to a foreign word $f_i$ according to the
   alignment function $a(j) = i$. The formula to calculate this probability is:
   $$ \Pr(e, a|f) =
   \frac{\epsilon}{(l_f + 1)^{l_e}} \prod_{j=1}^{l_e}t(e_j|f_{a(j)})$$
   where the leading fraction is used to normalize the distribution so
   all probabilities are between 0 and 1, and $t(e_j|f_{a(j)})$ is the
   probability of a given English word being translated into its
   aligned foreign word, from our earlier lexical translation
   probabilities. }
 \paragraph{}{Higher IBM Models increase in complexity and improve
   upon the first model's flaws; in short, they are \cite{smt}:}
 \begin{itemize}
 \item Model 1: lexical translation, as discussed above.
 \item Model 2: adds an explicit alignment model, which allows for a
   more statistically likely alignment.
 \item Model 3: adds a fertility model, which estimates the amount of
   output words that a foreign word `produces', allowing for input
   words to be dropped (eg. flavoring particles), and a distortion
   probability, which allows for more reordering in
   translation.\footnote{In practice, the Model 3 is sufficient for
     modern uses: using any models after this is superfluous.}
 \item Model 4: adds a relative alignment model, which fixes issues
   with Model 3's distortion model by looking at relative distortions
   instead.
 \item Model 5: fixes issues with previous models where multiple
   output words could be placed in the same positions, which wasted
   probability mass.
 \end{itemize}


 \paragraph{Phrase-based}{Phrase-based models are widely considered as
   state-of-the-art today \cite{smt} and outperform word-based models
   by a large margin \cite{koehn}. As the name suggests, they are
   based on phrases (short sequences of words), and processing is done
   on these phrases. The motivation behind this is clear to speakers
   of foreign languages: in many cases, a word in one language cannot
   be easily translated to another language because it has
   phrase-specific meanings. Phrase-based translation segments an
   input sentence into phrases, translates those phrases, and reorders
   them as needed. This has multiple advantages compared to the
   word-based model: it allows for one-to-many and many-to-one
   mappings, resolves translation ambiguities, and is conceptually
   much simpler. \cite{smt}}
 \paragraph{}{We start with exactly the same base equation as for word
   based models: that is, we look to maximize
   $$ \Pr(e|f) = \Pr(f|e)\Pr(e)$$ However, for the phrase-based model,
   we further decompose $\Pr(f|e)$ into phrase translations\cite{smt}:
   $$ \Pr(\bar{f}_1^I | \bar{e}_1^I)  =
   \prod_{i=1}^j \phi(\bar{f}_i | \bar{e}_i) \ d(\text{start}_i -
   \text{end}_{i-1} - 1)$$ We break up each foreign sentence $f$ into
   some $I$ phrases $\bar{f}_i$, and the same for each English
   sentence; each segmentation of phrases is equally likely. $\phi$ is
   a mapping of the probability of each English phrase being
   translated into a foreign sentence. The distance model $d$ is used
   to reorder the phrases.}
 \paragraph{}{Splitting a sentence into phrases can be done using a
   word alignment, like those seen in the earlier word-based
   translation section. We collect all aligned phrase pairs that are
   consistent with the word alignment. A phrase pair ($\bar{f}$,
   $\bar{e}$) is {\it consistent} with an alignment $A$ if all words
   in $\bar{f}$ with alignment points in $A$ have corresponding
   alignment points with words in $\bar{e}$, and vice versa: as a
   result, the words in a consistent phrase pair are only aligned with
   each other, and not to words outside that phrase
   pair.\footnote{This can leave us with phrases which are
     non-intuitive, for example `house the`: interestingly, pruning
     out these non-intuitive, syntactically incorrect pairs actually
     can decrease the performance of a model.\cite{koehn}} We can then
   estimate the phrase translation probability distribution in much
   the same way as we estimate $n$-gram probabilities, as
   $$ \phi(\bar{f} | \bar{e}) =
   \frac{\text{count}(f, \bar{e})}
   {\sum_{\bar{f}}\text{count}(\bar{f}, \bar{e})}$$ }
 \paragraph{}{The reordering of our English phrases, relative to the
   input foreign phrases, is done by the relative distortion
   probability function $d(s_i - e_{i -1})$, where $s_i$ is the start
   position, in terms of words, of the foreign phrase which was
   translated to the $i$th English phrase, and $e_{i-1}$ is the end
   position of the foreign phrase translated to the ($i-1$)th English
   phrase. For simplicity, $d$ is modeled by an exponentially decaying
   cost function, $d(x) \approx \alpha^{|x|}$, for some value of
   $\alpha$, $ 0 \leq \alpha \leq 1$, so larger distortions are much
   less probable, although some work has been done on modeling
   distortion using joint probability distributions \cite{wong}.}

%%% Local Variables:
%%% mode: latex
%%% TeX-master: "../Report"
%%% End:

  \vfill\null\columnbreak       
  \section*{Language model}
  \paragraph{Language model}{With a language model, we can assign a
  probability to a sequence of words. Intuitively, the sentence `the
  house is small' \cite{smt} should be more likely than `small the is
  house'.\footnote{However, to quote Chomsky: ``It must be recognized
    that the notion `probability of a sentence' is an entirely useless
    one''. The probability of a sentence has no intrinsic linguistic
    meaning, because humans do not produce language on a statistical
    basis.} We can use a statistical language model to accurately
  describe just how much more likely it is. To build our language
  model, we take a single-language corpus and use $n$-grams by way of
  the Markov assumption\cite{markov}, in a similar way to Weaver's
  original suggestions.}
\paragraph{}{Let's say that we want to estimate the probability of a
   sequence of words, $W = w_1, w_2, \dots w_n$. The na{\"i}ve
   approach to this would be to take a sufficiently large corpus,
   count how many times $W$ appears in it, and divide by the total
   number of $n$-word sequences to get a probability for $W$: however,
   even for an extremely large corpus, it is very unlikely that $W$
   will have appeared at all in its exact form, making probability
   distributions extremely sparse. What we can do instead is use the
   chain rule to decompose the probability of $W$\cite{smt}:
   $$ \Pr(w_1, w_2, \dots w_n) = \Pr(w_n | w_1 \dots w_{n-1})$$ This
   is still cumbersome to calculate, so we can restrict our
   calculations to only a history of words, that is
   $$ \Pr(w_n | w_1 \dots w_{n-1}) \approx \Pr(w_n|w_{n-m} \dots
   w_{n-1})$$ We use the Markov assumption here by saying that only a
   limited number of previous words affect the probability of the
   $n$th word.\footnote{While this assumption is technically wrong,
     it is a convenient simplification, because shorter word histories
     have less sparse probability distributions.} Most commonly,
   language models will restrict their search space to the previous
   two words (called {\it trigrams}, because they consider sets of
   three words) or just the previous word (correspondingly called {\it
     bigrams}). Expanding the search space for larger values of $n$ is
   usually a matter of diminishing returns. }
 \paragraph{}{If we therefore wanted to estimate the probability of a
   given trigram consisting of the sequence $T = w_1, w_2, w_3$, we would 
   count the amount of times we see the sequence $w_1, w_2, w_3$ in
   the corpus, and divide it by the amount of times we see just the
   sequence $w_1, w_2, x$ for all words $x$: that is, we divide the
   count of that specific trigram by the count of all trigrams which
   start with its history. Then by maximum likelihood estimation
   \cite{smt}, the probability of $T$ is therefore:
   $$ \Pr(T) = \frac{\text{count}(w_1, w_2, w_3)}
   {\sum_x \text{count} (w_1, w_2, x)}$$ \\The formula for a general
   $x$ is defined similarly. We can calculate this for all occurring
   $n$-grams in a corpus to give us an $n$-gram probability
   distribution for that corpus. }
 \paragraph{}{It is worth noting at this point that these resulting
   probabilities are not (and should not be!) used as they are:
   $n$-gram models are very sensitive to the training corpus
   \cite{jur}, and as such we have two issues: sparse data
   (probabilities for most $n$-grams will generally be quite low, by
   Zipf's law \cite{zipf}, and will be zero for $n$-grams not seen in
   the training corpus), and how to deal with out-of-vocabulary (OOV)
   tokens (any $n$-gram containing a word not seen in the training
   corpus will have a probability of 0). To deal with OOV tokens, we
   can either simply insert an `unknown' token (usually `UNK') into
   the training data, or we can replace the first occurrence of each
   vocabulary word with `UNK', and then calculate its probability
   distribution in the same manner as any other word. The best method
   of dealing with the sparse data issue is by using smoothing
   methods: these allow us to get better estimates for $n$-grams which
   occurred rarely in the training data, or not at all. There are
   numerous methods, which I will not go into here, including
   Kneser-Ney Smoothing and Good-Turing estimation. Backoff and
   interpolation methods may also be used\cite{jur}. }

%%% Local Variables:
%%% mode: latex
%%% TeX-master: "../Report"
%%% End:

  \section*{Preserving poetic qualities}
  \paragraph{}{The previous two hypotheses may improve poetic qualities
  of translations, but they may not necessarily preserve the poetic
  techniques that were used in the original poem. There are two
  possible ways to approach this:}
\begin{enumerate}
\item Hypothesis constraints, which can preserve such features as line
  length (syllable and word), rhyme, or meter. \cite{genzel}
\item Post-processing: in effect, building a translation system from
  non-poetic to poetic English.
\end{enumerate}

\paragraph{}{However, we are still faced with the problem of how to
  identify poetic techniques computationally. This is more difficult:
  nonetheless, some methods have already been developed to, for
  example, discover rhyme schemes.}
%%% Local Variables:
%%% mode: latex
%%% TeX-master: "Poster"
%%% End:

  \section*{Conclusion}
  \section*{Conclusions and future work}
\quot{Poetry is a sword of lightning, ever unsheathed, \\which
  consumes the scabbard that would contain it.}{Percy Bysshe Shelley}

\paragraph{}{This paper has discussed translation of poetry in a
  computational, statistically-based way. A discussion of the history
  of machine translation, the theory and practice of modern machine
  translation systems, and current and possible approachs to manual
  and computational translation, have all allowed me to explore the
  possibility of reaching high quality, fully automatic machine
  translation of poetry. }
\paragraph{}{The hypotheses I have discussed, of how current machine
  translation methods may be improved to create better translations of
  poetry, all lead nicely into future work. Each of the methods I have
  proposed --- training the language model with a custom-built corpus,
  altering the alignment model, and recovering poetic characteristics
  --- can easily be incorporated into existing machine translation
  systems such as Moses \cite{moses}. The next thing to do with this
  research would be to actually implement these methods, and compare
  the results of the new system with existing systems. One could also
  look at how other state of the art systems (for example neural
  machine translation systems) could be integrated with these
  hypotheses. }
\paragraph{}{In conclusion: translating poetry is hard. Both manually
  and computationally, there are many pitfalls which must be avoided
  to ensure a translation is faithful to the original and fluently
  written. I hope that with this report, the rarely-explored field of
  computational translation of poetry may find some new ideas. Perhaps
  in the future we may be able to update the quote which the
  introduction started with:\\}

\begin{minipage}{\linewidth}
\begin{displayquote}
  Poetry is that which is lost in translation, \\
  Unless we use a computational calculation.  \\  
\end{displayquote}
\end{minipage}

% \section*{Notes on this version of the paper}
% \paragraph{}{The theory section requires the following changes: a
%   gloss of how one would evaluate poetry (computationally and
%   manually) and a glossary. }
% \paragraph{}{The discussion section requires a further discussion of
%   using text-to-speech methods to identify word stress, and a
%   discussion of future work and implementations of my hypotheses. }

%%% Local Variables:
%%% mode: latex
%%% TeX-master: "Report"
%%% End:

  \begin{thebibliography}{10}
  \bibitem{nabokov} V. Nabokov. {\it Eugene Onegin: A Novel in Verse by
      Aleksandr Pushkin. Translated from the Russian}. 1964.
  \bibitem{chomsky} N. Chomsky. {\it Aspects of the Theory of
      Syntax}. 1965.
  \bibitem{bayes} T. Bayes \& R. Price. {\it An Essay towards solving a
      Problem in the Doctrine of Chance}. 1763.
  \bibitem{smt} P. Koehn. {\it Statistical Machine
      Translation}. 2010.
  \bibitem{genzel} D. Genzel \& J. Uszkoreit \& F. Och. {\it ``Poetic''
      Statistical Machine Translation: Rhyme and Meter}. 2010.
  \end{thebibliography}
\end{multicols}

\end{document}