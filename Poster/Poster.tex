\documentclass[a1, plainboxedsections, 13pt]{sciposter}
\usepackage[superscript]{cite}
\usepackage{multicol}
\usepackage{blindtext}
\usepackage{csquotes}
\usepackage{amsmath}
\renewcommand{\familydefault}{\rmdefault}

% \linespread{1.2}

\begin{document}
\title{{\Huge Towards Automated Translation of Poetry}}
\author{{\Large Michael A.}}
\maketitle
\conference{Exeter Mathematics Certificate 2017, Year 13 Project}

\begin{multicols}{3}
  \section*{Introduction}
  \paragraph{}{Translating poetry is hard. Without an intimate knowledge
  of the source and target language, a translation of any text will,
  without a doubt, be lacking. Translating poetry is even harder: and
  to do it computationally, almost impossible. This research project
  explores how we may improve current computational approaches to
  allow for better-quality translations of poetry.}
\paragraph{}{Why would we want to translate poetry? A couple of
  reasons come to mind, which specifically motivated my interest in
  this problem:}
\begin{itemize}
\item Foreign-language poems translated to English have historically
  built up the rich tapestry of English literature: where would we be
  without Dante, Virgil, Homer?
\item Machine translation could be used to translate poems to and from
  languages which have small, if any, translation communities.
\item A faithful translation requires the translator to have intimate
  familiarity with both the source and target languages: a computer
  could do one of these jobs, allowing for a lighter cognitive load on
  the translator and perhaps a better-quality translation.
\item Translated poems are often worth reading for their own merits,
  and it would be an interesting creative exercise to analyze machine
  translated poems.
\item The hope that a more computational, quantitative treatment of
  poetry and poetic analysis will encourage the more
  scientifically-minded to explore poetry and the beauty contained
  therein.
\end{itemize}
%%% Local Variables:
%%% mode: latex
%%% TeX-master: "Poster"
%%% End:

  \section*{Literary theories}
  \paragraph{}{Vladimir Nabokov, in the notes of his controversial
  translation of Aleksandr Pushkin's work `Eugene Onegin'
  \cite{nabokov}, argues that all translations of poetry will
  inevitably fall under three categories:}
\begin{description}
\item [Lexical] Translating the basic meaning of words and their
  order.
\item [Paraphrastic] A free version of the original: words and
  phrases are toyed with --- added, removed, or changed --- in order
  to conform to some form that the translator wishes.
\item [Literal] Translating as closely as possible the original, with
  the exact contextual meaning being preserved.
\end{description}

\paragraph{}{Computationally speaking, the easiest to implement is
  lexical translation. At most, it requires a good dictionary
  between the two chosen languages, along with a possible word
  realignment function. It will not preserve rhyme schemes,
  vocabulary choices, or poetic meter: but it may provide a useful
  gloss of the poem for someone only vaguely familiar with the
  original language.}
\paragraph{}{A literal approach, on the other hand, is
  currently impossible. It would require a `true' machine
  translation system: one which is fully able to understand the
  choices the author made, and is able to translate these choices
  into the target language. At this time, a literary translation of
  any poem is an AI-complete problem, and thus unsolvable until a
  strong AI system is built --- that is, a machine which is as
  emotionally intelligent as a person.}
\paragraph{}{However, a paraphrastic translation is entirely within
  the scope of existing machine translation systems. My research has
  concerned developing hypotheses which would allow current
  (phrase-based) translation algorithms to make a reasonable attempt
  at a paraphrastic translation. }
%%% Local Variables:
%%% mode: latex
%%% TeX-master: "Poster"
%%% End:

  \section*{Hypotheses}
  \paragraph{}{I have developed three hypotheses regarding how current
  phrase-based statistical approaches could be improved to better
  translate poetry:}
\begin{enumerate}
\item Training the language and alignment models on a poetic corpus
  improves poetic qualities of output translation.
\item Altering the sentence alignment model for poetic works will
  improve line-by-line translation quality.
\item Poetic characteristics can be preserved by constraints on the
  hypothesis space, or recovered by post-processing of the output
  translation.
\end{enumerate}

%%% Local Variables:
%%% mode: latex
%%% TeX-master: "Poster"
%%% End:

  \section*{Statistical methods}
  \paragraph{A statistical model of language}{All these
  rule-based, computational models of cognitive grammar are bound to
  be incomplete: simply put, we do not yet have enough knowledge of
  how the brain interprets and creates language --- what internal
  structures it uses, and so forth --- to be able to express these
  methods and structures computationally. So we compromise. Natural
  language processing and generation (of the kind needed in machine
  translation) doesn't require fully emulating human use of language:
  it only needs to approximate it, mimic it in such a way that it
  seems close to indistinguishable from the real thing. A good way of
  doing this is a probabilistic approach.}
\paragraph{}{The best way to explain the probabilistic model is in
  context: so let's take a look at the statistical model of machine
  translation. Say we want to translate a foreign sentence $f$ into an
  English sentence $e$. Then, by a probabilistic model\cite{smt}, we
  want our translation engine to maximize the probability that a given
  English sentence is the correct translation of the foreign sentence:
  that is, maximize $ \Pr(e|f) $. Intuitively, we can interpret this
  probability as the probability a translator will produce $e$ in the
  target language when presented with $f$ in the foreign
  language.\cite{ibm}}
\paragraph{}{Modeling this probability distribution is difficult: we
  can simplify the problem by applying Bayes' theorem\cite{bayes}: by
  Bayesian decomposition, we can rewrite $\Pr(e|f)$ as:
  $$ \Pr(e|f)=\frac{\Pr(e)\Pr(f|e)}{\Pr(f)}$$ Note that the
  denominator $\Pr(f)$ does not depend on $e$, and so is the same for
  all possible values of $e$: therefore we can just simplify our
  equation to be: $$ \Pr(e|f)= \Pr(e)\Pr(f|e)$$ This gives us two new
  variables to maximize: $\Pr(e)$, the {\it language model}, and
  $\Pr(f|e)$, the {\it alignment model}.}

%%% Local Variables:
%%% mode: latex
%%% TeX-master: "../Report"
%%% End:

  \section*{Alignment model}
  \paragraph{}{An alignment model is used to translate words or phrases
  from a foreign language into our target language. With a phrase-based
  translation system, we segment our input and output text into
  aligned phrases, and use these to build a probability
  distribution. We estimate this probability distribution by expanding $\Pr(f|e)$:
  $$\Pr(\bar{f}_1^I | \bar{e}_1^I) = \prod_{i=1}^j
  \phi(\bar{f}_i | \bar{e}_i) \ d(\text{start}_i - \text{end}_{i-1} -
  1) $$}
\paragraph{}{I hypothesize that we can separate poems into two
  different classes, which should have different alignment models
  applied to them, specifically with reference to sentence
  boundaries:}
\begin{enumerate}
\item Poems whose sentences can be treated in the same way as prose: standard sentence boundaries can be used to segment sentences.
\item Poems whose sentence boundaries are marked by line-endings,
  rather than punctuation.
\end{enumerate}

%%% Local Variables:
%%% mode: latex
%%% TeX-master: "Poster"
%%% End:

  \vfill\null\columnbreak       
  \section*{Language model}
  \paragraph{}{In machine translation, a language model is needed to
  make the output sound fluent in the target language. To estimate its
  probability distribution, we take a monolingual corpus and use
  $n$-grams to estimate the probability of word sequences.}
\paragraph{}{Language models are very sensitive to their training
  corpus: the training corpus should be in the same domain and on the
  same topic as the items to be translated. Therefore, I hypothesize
  that training the language model on a poetic corpus will increase
  the poetic qualities of a translation. Part of my research involved
  building a poetic corpus from the Project Gutenberg collection,
  which is large enough to feasibly use to train real models.}
%%% Local Variables:
%%% mode: latex
%%% TeX-master: "Poster"
%%% End:

  \section*{Preserving poetic qualities}
  \paragraph{}{The previous two hypotheses may improve poetic qualities
  of translations, but they may not necessarily preserve the poetic
  techniques that were used in the original poem. There are two
  possible ways to approach this:}
\begin{enumerate}
\item Hypothesis constraints, which can preserve such features as line
  length (syllable and word), rhyme, or meter. \cite{genzel}
\item Post-processing: in effect, building a translation system from
  non-poetic to poetic English.
\end{enumerate}

\paragraph{}{However, we are still faced with the problem of how to
  identify poetic techniques computationally. This is more difficult:
  nonetheless, some methods have already been developed to, for
  example, discover rhyme schemes.}
%%% Local Variables:
%%% mode: latex
%%% TeX-master: "Poster"
%%% End:

  \section*{Conclusion}
  \section*{Conclusions and future work}
\quot{Poetry is a sword of lightning, ever unsheathed, \\which
  consumes the scabbard that would contain it.}{Percy Bysshe Shelley}

\paragraph{}{This paper has discussed translation of poetry in a
  computational, statistically-based way. A discussion of the history
  of machine translation, the theory and practice of modern machine
  translation systems, and current and possible approachs to manual
  and computational translation, have all allowed me to explore the
  possibility of reaching high quality, fully automatic machine
  translation of poetry. }
\paragraph{}{The hypotheses I have discussed, of how current machine
  translation methods may be improved to create better translations of
  poetry, all lead nicely into future work. Each of the methods I have
  proposed --- training the language model with a custom-built corpus,
  altering the alignment model, and recovering poetic characteristics
  --- can easily be incorporated into existing machine translation
  systems such as Moses \cite{moses}. The next thing to do with this
  research would be to actually implement these methods, and compare
  the results of the new system with existing systems. One could also
  look at how other state of the art systems (for example neural
  machine translation systems) could be integrated with these
  hypotheses. }
\paragraph{}{In conclusion: translating poetry is hard. Both manually
  and computationally, there are many pitfalls which must be avoided
  to ensure a translation is faithful to the original and fluently
  written. I hope that with this report, the rarely-explored field of
  computational translation of poetry may find some new ideas. Perhaps
  in the future we may be able to update the quote which the
  introduction started with:\\}

\begin{minipage}{\linewidth}
\begin{displayquote}
  Poetry is that which is lost in translation, \\
  Unless we use a computational calculation.  \\  
\end{displayquote}
\end{minipage}

% \section*{Notes on this version of the paper}
% \paragraph{}{The theory section requires the following changes: a
%   gloss of how one would evaluate poetry (computationally and
%   manually) and a glossary. }
% \paragraph{}{The discussion section requires a further discussion of
%   using text-to-speech methods to identify word stress, and a
%   discussion of future work and implementations of my hypotheses. }

%%% Local Variables:
%%% mode: latex
%%% TeX-master: "Report"
%%% End:

  \begin{thebibliography}{10}
  \bibitem{nabokov} V. Nabokov. {\it Eugene Onegin: A Novel in Verse by
      Aleksandr Pushkin. Translated from the Russian}. 1964.
  \bibitem{chomsky} N. Chomsky. {\it Aspects of the Theory of
      Syntax}. 1965.
  \bibitem{bayes} T. Bayes \& R. Price. {\it An Essay towards solving a
      Problem in the Doctrine of Chance}. 1763.
  \bibitem{smt} P. Koehn. {\it Statistical Machine
      Translation}. 2010.
  \bibitem{genzel} D. Genzel \& J. Uszkoreit \& F. Och. {\it ``Poetic''
      Statistical Machine Translation: Rhyme and Meter}. 2010.
  \end{thebibliography}
\end{multicols}

\end{document}