\paragraph{}{It is hypothesized that humans have an internal grammar
  model, or a `universal grammar' \cite{chomsky}, which allows them to
  learn language in a way that other animals cannot. But exactly what
  this grammar is eludes researchers: so, try as we might, we cannot
  computationally emulate it. Instead, we can approximate human use of
  language using statistical methods.}
\paragraph{}{Say we want to translate a foreign sentence $f$ into an
  English sentence $e$. Then by a probabilistic model, we want to
  maximise the probability that a given English sentence is the
  correct translation of the foreign sentence: that is, maximise
  $\Pr(e|f)$.\cite{smt} Modelling this probability distribution is
  difficult, so we simplify the problem by applying Bayes' theorem:
  \cite{bayes}
  $$ \Pr(e|f)= \Pr(e)\Pr(f|e)$$ This gives us two variables to
  maximize: $\Pr(e)$, the {\it language model}, and $\Pr(f|e)$, the
  {\it alignment model}.}
%%% Local Variables:
%%% mode: latex
%%% TeX-master: "Poster"
%%% End:
